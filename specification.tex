\chapter{Design Specification}
\label{chap:specification}
This research is a study of the viability of leveraging augmented reality technology for simulation-based training purposes in the maritime sector. A working prototype was envisaged in order to access the user experience of such a system. This chapter contains the devised design solution - specifications that would be used to develop a prototype for evaluation. 

%to the ship handling training problem outlined in the previous chapter. 

Development of the actual prototype was based on a few functional requirements thought to be essential to the system (section \ref{sec:functionalreqs}). SCE method prescribes that functional requirements be linked to objectives of the system. The prototype that is developed is then tested for it's ability to meet desired objectives. Evaluation hinges on claims that are made regarding the system (section \ref{sec:claims}). Claims in this context are hypothesis whose truth value is determined during system evaluation. First, section \ref{sec:trainingcomparison} makes a comparison of three types of on-board training differing in their expected learning outcomes. Section \ref{sec:usecase} lists a basic use-case flow and alternate steps of system use.




\section{Design Scenario}

\textbf{“AugMan helps Michiel practice ship maneuvering and, improve depth perception.”}

Michiel, 25 years old, received the dynamic positioning (DP) certificate a short time ago. He has just taken up his first job as a dynamic positioning operator (DPO) on-board VOS BASE - a medium-size platform supply vessel. Fitted with Class 1 DP system, the vessel has been rented by Royal Dutch Shell to supply cargo to offshore oil platforms off the Dutch coast. Michiel is joined on the deck by captain Willem and chief mate Steve, both experienced seafarers capable of handling the ship as well as its DP equipment.

He lacks manual vessel manoeuvring skills. Though familiar with bridge equipment, he has little experience using it on an actual vessel. Manoeuvring a large vessel in close range of offshore platforms causes him anxiety. He feels unprepared for an emergency such as failure of one of the DP system's components effectively rendering it unusable. Should this situation arise during a load/unload operation, an operator needs to maintain the vessel's position and heading manually - at least until its safe to stop the operation. Following this, the vessel will be steered away outside the 500m safety zone of the oil rig. With a dysfunctional DP system, either joystick manoeuvring or manual vessel handling will have to be engaged; both of which he is not adequately trained for. 

\section{Functional Requirements}
\label{sec:functionalreqs}
%The prototype would be used to test feasibility of the augmented-reality ship maneuvering training method described here.
This section lists the functional requirements that were drawn for a prototype application. Requirements were based primarily on specifications of augmented reality put forward by \cite{azuma1997survey}. They are tailored to suit the operational demands of ship manoeuvring training. Accordingly, functional requirements of the prototype are listed below.\\

The system shall be able to:
\begin{enumerate}[itemsep=0.01em]
\item \textbf{Present 3D images} of objects such as ships, oil platforms, buoys and landscape features to create the augmented environment.
\item \textbf{Register virtual objects} in user’s physical space.
%to create a convincing feeling of the object’s presence.
\item \textbf{Generate projections} of virtual object to be consistent with user position.
\item Provide exercises to \textbf{accommodate various learning goals}. 
\item \textbf{Provide manoeuvring instructions} adapted to user's manoeuvring proficiency.
\end{enumerate}

\section{Claims}
\label{sec:claims}
\begin{enumerate}[itemsep=0.01em]
	\item Presenting user with 3D images helps create a convincing feeling of presence of an object in user’s reality. 
	\item Registering the virtual object in a fixed space creates a convincing feeling of presence.
%	required for depth perception, so that size can be compared with that of the surroundings. 
	\item Generating view-specific projections of virtual object based on user position is essential to leverage parallax effect which provides depth cues.
	\item Manoeuvring instructions, help automate the task of coaching. 
	\item Different types of manoeuvring exercises makes the system useful to a wide range of users of varying levels of manoeuvring skills.
\end{enumerate}

\section{Use Case}
\label{sec:usecase}
\subsection{Manoeuvre Training}
\fbox{
	\parbox{0.97\linewidth}{
		\begin{description}[labelwidth=3.5cm]
			\item [Actors:] DP Operator, AugMan, Captain/Instructor.
			\item [Circumstance:] DP operator is idle on an offshore supply vessel.
			\item [Precondition:] Vessel is not undertaking any operation.
			\item [Post condition:] DP operator has improved manoeuvring skills.
			\item [Method:] Practice manoeuvring in augmented reality.
		\end{description}
}}

%	 It is idle at sea and is not in the midst of traffic from other seafaring vessels.

\subsection{Basic Flow}
\begin{enumerate}[itemsep=.01em]
	\item Operator wears AR device and starts the application. 
	\item AugMan presents a list manoeuvring exercises and asks user to choose.
	\item Operator chooses an exercise and orients himself (including head position) in the direction in which the virtual object/augmented environment should appear.
	\item Operator indicates he is ready to start the exercise.
	\item AugMan brings up the augmented reality environment and asks the operator to set vessel controls to neutral position before starting the exercise.
	\item After making sure that the vessel controls are in neutral position, operator starts the exercise. 
	\item AugMan receives position changes from the vessel and renders visuals of the virtual environment accordingly.
	\item Operator practices manoeuvring the vessel in the virtual environment using visuals from the AR device as reference.
	\item AugMan provides a review and feedback of the manoeuvring performance at the end of the exercise.
\end{enumerate}

\subsubsection{Alternative steps (Beginner Level)}

Application provides direction and manoeuvring hints throughout the exercise.

\begin{enumerate}[noitemsep]
	\item Operator wears Ar device and starts the application. 
	\item AugMan presents a list of \textbf{low-difficulty manoeuvring exercises} available and asks user to choose.
	\item Operator chooses an exercise and orients himself (including head position) in the direction in which the virtual object/augmented environment should appear.
	\item Operator indicates he is ready to start the exercise.
	\item AugMan brings up the augmented reality environment and asks the operator to set vessel controls to neutral position before starting the exercise.
	\item After ensuring that the vessel controls are in neutral position, operator starts the exercise. 
	\item AugMan receives position changes from the vessel and renders visuals of the virtual environment accordingly.
	\item Operator practices manoeuvring the vessel in the virtual environment using visuals from the AR device as reference.
	\item Application provides \textbf{helpful hints} for manoeuvring such as the directional pointers and controls to be engaged \textbf{throughout the exercise}.
	\item AugMan provides a review and feedback of the manoeuvring performance at the end of the exercise.
\end{enumerate}

\subsubsection{Alternative steps (Intermediate Level)}

Application provides maneuvering hints during the exercise on request by user.

\begin{enumerate}[noitemsep]
	\item Operator wears AR device and starts the application. 
	\item AugMan presents a list of \textbf{medium-difficulty maneuvering exercises} available and asks user to choose.
	\item Operator chooses an exercise and orients himself (including head position) in the direction in which the virtual object/augmented environment should appear.
	\item Operator indicates he is ready to start the exercise.
	\item AugMan brings up the augmented reality environment and asks the operator to set vessel controls to neutral position before starting the exercise.
	\item Having ensured that the vessel controls are in neutral position, operator starts the exercise. 
	\item AugMan receives position changes from the vessel and renders visuals of the virtual environment accordingly.
	\item Operator practices maneuvering the vessel in the virtual environment using visuals from the AR device as reference.
	\item Application provides \textbf{helpful hints} for maneuvering such as the controls to be engaged during the exercise \textbf{on need basis}.
	\item AugMan provides a review and feedback of the maneuvering performance at the end of the exercise.
\end{enumerate}


\subsubsection{Alternative steps (Expert Level)}

Application does not provide maneuvering hints during the exercise.

\begin{enumerate}[noitemsep]
	\item Operator wears AR device and starts the application. 
	\item AugMan presents a list of \textbf{high-difficulty maneuvering exercises} available and asks user to choose.
	\item Operator chooses an exercise and orients himself (including head position) in the direction in which the virtual object/augmented environment should appear.
	\item Operator indicates he is ready to start the exercise.
	\item AugMan brings up the augmented reality environment and asks the operator to set vessel controls to neutral position before starting the exercise.
	\item Having ensured that the vessel controls are in neutral position, operator starts the exercise. 
	\item AugMan receives position changes from the vessel and renders visuals of the virtual environment accordingly.
	\item Operator practices maneuvering the vessel in the virtual environment using visuals from the AR device as reference.
	\item Application provides \textbf{no maneuvering hints} during the exercise.
	\item AugMan provides a review and feedback of the maneuvering performance at the end of the exercise.
\end{enumerate}

\subsection{Ontology}
\begin{description}
\item[Augman] - augmented reality prototype consisting of see-through glasses and a head tracking system capable of registering objects a predetermined location in space
%built using TrackIR that runs on a PC, and a software program that streams head position information to the smartglasses.
%TrackIR - Motion tracking device that tracks head motions with up to 6 degrees of freedom of movement. Conceived originally as a game controller that enables handsfree view control, it uses a video camera that observes infrared light that is reflected/emitted by markers worn by the user.
\item[See-through glasses] - Wearable see-through glasses that are capable of rendering 3D graphics making it possible to create augmented reality environments which leaves the wearer’s view of the outside world intact while rendering virtual objects in the same visual field.
\item[DP Operation trainee] - Dynamic positioning operator with limited manual vessel handling experience.
\item[DP Vessel] - A vessel with dynamic positioning system installed on it that can also be operated in manual mode wherein the automatic vessel position control systems can be disabled for training purposes.
\item[Training supervisor] - A qualified and experienced officer-of-the-watch who is capable of manual vessel control that instructs trainees on learning goals, provides feedback on performance and can take control of the vessel if necessary.

\end{description}