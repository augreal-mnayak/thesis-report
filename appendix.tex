\chapter{Appendix}

\section{Photorealism}
\begin{table}[h]
	\centering
	\begin{tabular}{L{1.5cm}L{9cm}@{}}
		\toprule
		\textbf{Level} & \textbf{Attributes} \\ 
		\midrule
		High & \vspace{-11pt} \begin{itemize}[leftmargin=*,topsep=0pt,align=left,itemsep=0.5pt,after=]
			\renewcommand{\labelitemi}{\scalebox{.9}{\tiny\listsymb}}
			\item Occlusion is necessary to induce sense of depth in scene.
			\item Wide field of view is required to create illusion of large object in the vicinity.	
			\item Virtual object is the main object of focus in scene. 
			%			Inconsistency in 3D view of augmented reality can hamper task performance from visual fatigue due to accomodation-vergence conflict.
		\end{itemize}\\
		\hline
		Medium & \vspace{-11pt} \begin{itemize}[leftmargin=*,topsep=0pt,align=left,itemsep=0.25pt,after=]
			\renewcommand{\labelitemi}{\scalebox{.9}{\tiny\listsymb}}
			\item Occlusion is beneficial to the scene, transparency is tolerable to an extent.
			\item Wide field of view is not essential to the scene. 
			\item Virtual object is not the main object of focus in scene.
		\end{itemize}\\
		\hline
		Low & \vspace{-11pt} \begin{itemize}[leftmargin=*,topsep=0pt,align=left,itemsep=0.25pt,after=]
			\renewcommand{\labelitemi}{\scalebox{.9}{\tiny\listsymb}}
			\item Depth cues are not essential to the scene.
			\item Wide field of view is not essential to the scene.
			\item Virtual object is not the main object of focus in scene.
		\end{itemize}\\
		\bottomrule
	\end{tabular}
	\caption{Levels of Photorealism}
	\label{tab:trainingoptions}
\end{table}

\section{On-board Training Possibilities}

\begin{table}[h]
	\centering
	\begin{tabular}{@{}p{2.2cm}|p{3cm}|p{3cm}|p{3cm}|@{}}
		\toprule
		\multicolumn{1}{c|}{ } & \multicolumn{1}{P{3cm}|}{\textbf{Manoeuvre Training}} & \multicolumn{1}{P{3cm}|}{\textbf{Navigation Training}} & \multicolumn{1}{P{3cm}|}{\textbf{Emergency Response}} \\ 
		\hline
		Learning Objectives & Vessel handling in navigationally challenging situations & Long-distance maritime navigation & Emergency response procedures and teamwork skills \\
		\hline
		Scenarios & Position keeping, In-place turning, Berthing
%		\vspace{-2mm} \begin{itemize}[leftmargin=*,topsep=0pt,align=left,itemsep=0.25pt,after=]
%			\renewcommand{\labelitemi}{\scalebox{.9}{\tiny\listsymb}}
%			\item Position keeping  
%			\item In-place turning
%			\item Berthing  
%		\end{itemize}
%		\vspace{-2mm}
		& 
		Path planning, Navigating port entrances
%\vspace{-2mm} \begin{itemize}[leftmargin=*,topsep=0pt,align=left,itemsep=0.25pt,after=]
%			\renewcommand{\labelitemi}{\scalebox{.9}{\tiny\listsymb}}
%			\item Path planning 
%			\item Navigate port entrances
%		\end{itemize} 
%		\vspace{-2mm}
		& 
		Fire in engine room, Water flooding in lower deck, Man over-board
%		\vspace{-2mm} \begin{itemize}[leftmargin=*,topsep=0pt,align=left,itemsep=0.25pt,after=]
%			\renewcommand{\labelitemi}{\scalebox{.9}{\tiny\listsymb}}
%			\item Fire in engine room 
%			\item Water flooding in lower deck
%			\item Man over-board
%		\end{itemize} \vspace{-2mm}
	\\
		\hline
		Competences Gained & \vspace{-2mm}\begin{itemize}[leftmargin=*,topsep=0pt,align=left,itemsep=0.25pt,after=]
			\renewcommand{\labelitemi}{\scalebox{.9}{\tiny\listsymb}}
			\item Bridge equipment operation
			\item Vessel movement intuition
			\item Depth estimation  
		\end{itemize}
		& \vspace{-2mm}\begin{itemize}[leftmargin=*,topsep=0pt,align=left,itemsep=0.25pt,after=]
			\renewcommand{\labelitemi}{\scalebox{.9}{\tiny\listsymb}} 
			\item Route planning
			\item Identify, use maritime navigation aids
		\end{itemize}
		& \vspace{-2mm}\begin{itemize}[leftmargin=*,topsep=0pt,align=left,itemsep=0.25pt,after=]
			\renewcommand{\labelitemi}{\scalebox{.9}{\tiny\listsymb}}
			\item Emergency severity assessment 
			\item Teamwork, communication 
		\end{itemize}  \\
		\hline
		Equipments Required & Radar, ARPA, ECDIS, Fiducial markers in bridge & Radar, ECDIS, ARPA, AIS, Markers in bridge & Fiducial markers in engine room, deck, etc. \\
		\hline
		Crew Requirement & 
		\vspace{-2mm} \begin{itemize}[leftmargin=*,topsep=0pt,align=left,itemsep=0.25pt,after=]
			\renewcommand{\labelitemi}{\scalebox{.9}{\tiny\listsymb}}
			\item Trainee 
			\item Instructor
			\item Officer of the watch
		\end{itemize}
		&\vspace{-2mm} \begin{itemize}[leftmargin=*,topsep=0pt,align=left,itemsep=0.25pt,after=]
			\renewcommand{\labelitemi}{\scalebox{.9}{\tiny\listsymb}}
			\item Trainee 
			\item Instructor
			\item Officer of the watch
		\end{itemize}
		& \vspace{-2mm} \begin{itemize}[leftmargin=*,topsep=0pt,align=left,itemsep=0.25pt,after=]
			\renewcommand{\labelitemi}{\scalebox{.9}{\tiny\listsymb}}
			\item Full crew / part of crew located at emergency site
		\end{itemize} \\
		%Trainee, OOW, Captain/Instructor & Trainee, OOW, Captain/Instructor & Full crew / part of crew located at emergency site \\ 
		\bottomrule
	\end{tabular}
\caption{Comparison of on-board training options}
\label{tab:trainingoptions}
\end{table}
\newpage
\section{Optical vs. Video See-Through Display}
\begin{table}[h]
	\centering
	\begin{tabular}{p{3.3cm}|P{3.7cm}|P{4.2cm}@{}}
		\toprule
		\multicolumn{1}{P{3.3cm}|}{} & 
		\multicolumn{1}{P{3.7cm}|}{\textbf{Optical}} & 
		\multicolumn{1}{P{4.2cm}}{\textbf{Video}} \\ 
		\midrule
		Peripheral FOV (horizontal)      & 180 degrees                                                            & 110 degrees \\
		\hline
		Time lag - real world view                & 0ms                                                                    & \textgreater 16ms                                                                                              \\ 
		\hline
		Digital display FOV (horizontal) & 20-40 degrees                                                          & \textgreater 90 degrees                                                                                        \\
		\hline
		Real world view                 & Largely undistorted                                                    & Camera and display dependent                                                                                   \\
		\hline
		Simplicity                                 & Process one stream  (virtual images)                            & Process two streams (camera feed \& virtual)                                                         \\
		\hline
		%\todo{add comparison of calibration and registration}
		%Calibration                                & Need to calibrate for user's unique facial geometry and facial acuity. & Availability of single digital composite of real and virtual images allows use of computer vision techniques. \\
		
		Resolution                                 & Partially display dependent                                            & Fully Camera and display dependent                                                                             \\
		\hline
		Focus and contrast                         & Hard to blend virtual object into real scene                           & Lesser contrast issues (limited by camera's dynamic response )                                       \\
		\hline
		Occlusion                                  & Challenging to achieve full occlusion									& Occlusion is possible due to full control over displayed content \\ 
		\bottomrule 
	\end{tabular}
	\caption{Comparison of see-through head-mounted displays}
	\label{tab:comparehmds}
\end{table}

\section{Expert Interview Questions}
\begin{enumerate}
	\item How effective is simulator based training to learn vessel maneuvering?
	\begin{enumerate}
		\item Is a trainee capable of maneuvering immediately after simulator based training?
		\item If not, what is missing?
		\item How is manoeuver training conducted onboard? 
		\item Why is there no certification system for tug masters? 
	\end{enumerate}

	\item What is your opinion of SMSC's onboard training system? 
	\begin{enumerate}
		\item Pros and Cons?
		\item Is it necessary to simulate the bridge equipment?
		\item Which bridge equipment is important to create the virtual platform illusion? 
	\end{enumerate}

	\item Tugs, Offshore Supply Vessels - PSV, AHTS, ERRV, Construction vessel, Diving support vessel
	\begin{enumerate}
		\item Which of the above has the most idle time on average?
		\item How much idle time do they have every day on average?
		\item What do they do during idle time? 
		\item Which of them is suitable for onboard training (based on idle time, safety, other criteria)?
		\item Where can they do onboard training?
		\item Is there an officer on the lookout onboard OSVs? Is he always on watch? 
	\end{enumerate}

	\item Are there DPOs with little manual vessel maneuvering experience? 
	\begin{enumerate}
		\item Would they like to learn vessel maneuvering? 
		\item Is over-reliance on DP systems a matter of concern in the maritime industry?
	\end{enumerate}

	\item Is there a familiarization procedure in place for captains taking over new vessels? 
	\begin{enumerate}
		\item How do captains familiarize themselves to new vessels? 
		\item How can the level of familiarization be measured?
	\end{enumerate}

	\item Are there differences in skill between a pilot and a captain?
	\begin{enumerate}
		\item Can this be measured? How?
		\item What are the technical skills of a skilled maritime navigator? (e.g. certain maneuvers like maintain position, turning in place, etc.?)
	\end{enumerate}
\end{enumerate}
